%!TEX root = ../main.tex
% -*- root: ../main.tex -*-
\subsection{A3-E Mobile Middleware}

In order to assess the feasibility of the presented model and to validate it with an experimental evaluation we implemented a working prototype of the \textit{mobile middleware}. As described in Section~\ref{sec:proposal} this component interacts with  domains that could be either discoverable using a DNS-like mechanism or advertisement. The prototype is one of the possible materialization of the A3-E model, therefore it embraces the computing continuum: client applications can invoke functions without knowing where they will be actually executed (either locally, in one of the surrounding edge servers or in the cloud). The selection algorithm is based on the functions requirements, the domains availability and a multi-objective decision algorithm that will be described in the following. The implementation\footnote{Documentation and source code are available at \url{https://github.com/gioenn/a3e-android}} was written in Java for the Android operating system but could be easily generalized to any mobile platforms. 
\begin{figure}[tbp]
	\includegraphics[width=0.9\textwidth]{figs/a3e-mobile-prototype}
	\caption{Mobile Middleware Architecture}
	\label{fig:mobile-prototype}
\end{figure}

Figure~\ref{fig:mobile-prototype} shows the architecture of the client middleware using an UML-like notation. Client mobile applications can support the continuum by simply interacting with two components:  \textit{A3EFunction}s and \textit{A3EFacade}. The former abstracts the actual functions to be executed, while the latter is the main interface by which it is possible to first register and then execute functions. An \textit{A3EFunction} is identified by a unique name that corresponds to the name of the function asset that must be communicated to the continuum domains to be first acquired and then executed. Moreover each function must declared with a set of non-functional QoS requirements that are organized in three types: 

\begin{itemize}
	\item \textit{Location Requirement}s are constrains over the continuum. A function can declare where it could be executed choosing a combination of three values: \textit{LOCAL}, \textit{EDGE} and \textit{CLOUD}. By default an \textit{A3EFunction} supports all three domain types but one can implement a function that, for example, cannot be executed locally thus only the \textit{EDGE} and \textit{CLOUD} requirements should be added.    
	\item \textit{Latency Requirement} expresses how important for a function is to have a low networking latency. The default value is \textit{LOW} since the main motivation of the work is to support low-latency applications. An \textit{A3EFunction}  can  also state that this requirement should be \textit{ANY} or \textit{VERY LOW}. The lower the latency requested the more the networking latency will be considered important in the the domain selection procedure.
	\item \textit{Computational Requirement} defines how relevant for a function is to have a fast computing processing. The predefined value is \textit{FAST}, since, again, the main targets of the approach are applications that requires fast request/response interactions. Similarly to the latency requirement two additional values are available:  \textit{ANY} or \textit{VERY FAST}. The higher  processing power is requested the more this metric will be considered important during the domain selection phase.
\end{itemize}

An \textit{A3EFunction} that support the \textit{LOCAL} location requirement should also define a local \textit{InvocationResolver}. As we are going to discuss later in this section, invocation resolvers deal with the invocation technology \textit{heterogenity}  of the continuum. For what regards the local domain we currently support the execution of native Java code and Javascript functions (that could be imported in the project as standard \textit{.js} files). For this purpose we created two \textit{InvocationResolver}s that can execute respectively Java or Javascript functions if the local domain is selected.  Finally before being able to execute it, the function must be registered using the \textit{A3EFacade}.

The \textit{A3EFacade} wraps the \textit{A3ELoopManager} that stores all the register \textit{A3EFunctions} and manages a MAPE-insipired control loop for the discovery, identification and selection of domains corresponding to the \textit{Awareness}, \textit{Acquisition}, and \textit{Allocation} phases of the A3E model. The loop manager  consists of three main components called \textit{DiscoveryManager}, \textit{IdentificationManager} and \textit{SelectionManager}. 

The \textit{DiscoveryManager} explores the surrounding environment looking for available domains. An  \textit{A3EDomain}s could be either \textit{static} or \textit{dynamic}. Static domains are added to the discovery manager at launch time, meaning that it is known a-priori that they will be available. Example of this are cloud and local domains. On contrary dynamic domains can be found only at runtime with appropriate technological protocols (such as DNS and advertising). Edge domains are not known a-priori thus they are considered dynamic. Static domains can be added by the client application using the \textit{registerDomain} operation provided by the \textit{A3EFacade} while dynamic domains are automatically discovered by the \textit{DiscoveryManager}.

 \textit{A3EDomain}s  are identified by an \textit{host} name, that is a unique network identifier by which it is possible to interact with the domain using a network. Moreover, as shown in the figure, a domain is considered \textit{Pingable} that means that it is possible to check for its availability and measure its networking latency (in milliseconds). Each domain satisfy a \textit{LocationRequirement}, intuitively local domains satisfies the \textit{LOCAL} requirement, while the edge and cloud ones the \textit{EDGE} and \textit{CLOUD} respectively.  A domain also provides three additional operations: one to execute a function within the domain, one to retrieve the networking latency (its value is lazily update after a \textit{ping}) and one to obtain its computational power. Finally a domain also uses an \textit{InvocationResolver} to actually invoke the function. 
 
Therefore, in addition to find new domains, the \textit{DiscoveryManager} also checks for their availability pinging, at each control loop iteration, all the discovered domains retrieving their respective network latency. The output of this process is the list of the available domains. After that the \textit{IdentificationManager} is activated. This component initializes the identification phase for each new available domain (e.g., domains discovered for the first time). For each new domain first it checks which registered functions can be executed on it (i.e., fulfillment of the location requirement), than it communicates those functions (eventually triggering the allocation phase in the domain) and finally it retrieves its computational power. For this metric, in the current implementation, we are considering a fixed score ranging from $1$ to $5$ where the local domain is set to $1$, the edge domains to $4$ and the cloud ones to $5$. Labeling computational power is also common in the cloud where different tiers of virtual machines are available (e.g., micro, small, large). Note that in the continuum we label domains and not single machines and since the cloud has an infinite degree of scalability it gets the maximum score, independently of the machines in use. More sophisticated approaches, such as supporting dynamic scores that change according to the saturation of the domain or, in case of the local one, with respect to the battery level, are considered future works.  

\begin{algorithm}[b]
	\caption{A3E Selection Algorithm}
	\label{alg:selection}
	\begin{algorithmic}[1]
		
		\Function{selectDomain}{A3EFunction $function$, A3EDomain[] $identifiedDomains$}
		\State$scoreRange \gets 5$
		\State $maxLatency \gets \Call{computeMaximumLatency}{identifiedDomains}$
		\State $maxCpuPower \gets \Call{computeMaximumComputationalPower}{identifiedDomains}$
		\State $latencyWeight \gets function.getLatencyRequirement()$ 
		\State $cpuPowerWeight \gets function.getComputationalPowerRequirement()$ 
		\State $maxScore \gets 0$
		\State $selectedDomain \gets null$
		\ForAll{$domain \in identifiedDomains$ } 
		\State $latency \gets domain.getLatency()$ 
		\State $cpuPower \gets function.getComputationalPower()$ 
		\State $latencyScore \gets latencyWeight*((scoreRange-1)*(1 - latency/maxLatency)+1)$ 
		\State $cpuPowerScore \gets cpuPowerWeight*(scoreRange*(cpuPower/maxCpuPower))$
		\State $score \gets (latencyScore + cpuPowerScore) / (latencyWeight + cpuPowerWeight)$
		\If{$score \geq maxScore$} 
		\State $maxScore \gets score$
		\State $selectedDomain \gets domain$
		\EndIf
		\EndFor 
		\State \Return $selectedDomain$
		\EndFunction
	\end{algorithmic}
\end{algorithm}

At the end of the identification phase, the networking latency and computational power of each domain is stored and the domain is considered ready to, eventually, execute functions (i.e., proactive approach). After that the \textit{SelectionManager} starts the selection process for each function. 
Algorithm~\ref{alg:selection} shows the procedure of the selection process. The procedure is activated for each registered function with the gathered identified domains (line $1$). The algorithm consists in computing a score ranging from $0$ to $5$ (line $2$) for each domain and select the one with the highest one. The current implementation considers as quality metrics the networking latency and the computational power. The first step of the selection consists in computing the maximum latency and the maximum computational power (line $3$ and $4$) among the identified domains. This data, as already mentioned, were gathered during the previous phases. Then the weights of latency and computational power are retrieved as shown at lines $5$ and $6$. These weights correspond to the values associated to the \textit{LatencyRequirement} and \textit{ComputationalPowerRequirement} of the function. The \textit{ANY} value corresponds to a weight of $0$, a latency requirement of \textit{LOW} and a computational power requirement of \textit{FAST} correspond to a weight equal to $1$, while weight equals to $2$ is considered for both latency \textit{VERY LOW} and computational power \textit{VERY FAST}.

For each domain we then compute the overall score as shown from line  $9$ to $14$. The latency score is computed by normalizing the latency value retrieved at line $10$ with the maximum latency previously computed. The normalized value ranges from $0$ to $1$ and the higher this value the higher the latency. Since we want an higher score for lower latencies we compute the complement of this value and add $1$ to avoid scores equals to $0$. Finally the latency score is computed to be in the range of $5$ and multiplied by the function's latency weight (line $12$). The computational power score is computed by normalizing the domain computational power retrieved at line $11$ with maximum value across the identified domains. Then the score for this metric is computed to be in the range of $5$ and multiplied by its weight (line $13$). Finally the overall score is computed by performing an weighted average between the scores obtained by the domains for the two QoS metrics.

Two considerations must be added for this algorithm. First, the procedure is an instantiation of the SMART approach~\cite{Olson1996} where multiple competing QoS objectives are considered in the decision process following the formula
\begin{equation}
Smart(c) = \frac{\sum_{i=1}^{n} QoSAtrribute_i(c)*weight_i)}{\sum_{i=1}^{n}weight_i} 
\end{equation}
where, in the context of our approach, $c$ is a domain, $n$ is equal to $2$ and the QoS attributes values are networking latency and computational processing time.
Second, \textbf{the edge domains have the highest chances to be selected} (if available) since they usually combine a very low network latency and a medium to high computational power. 

After this steps each function has its own selected domain that satisfies at best its requirements. At a control interval customizable by the user (default is $2$ seconds) the control loop restarts to look at possible new domains, the updated networking latency and, eventually, better selections.

The last phase to be described is the \textit{Engagement} phase where the actual function execution takes place. The client application can execute a function using the \textit{execute} operation of the  \textit{A3EFacade}. In addition to the \textit{A3EFunction} the operation expects also \textit{payload} (i.e., the function argument) that could be an object of any Java type and a callback since the execution is always asynchronous. The \textit{fa\c{c}ade} will then retrieve from the \textit{A3ELoopManager} the last selected \textit{A3EDomain} for the function. Since the control loop runs in a separated thread, the execution will not be affected if the selected domain changes during the invocation. After retrieving the best domain from the continuum the \textit{A3EFacade} will execute the function by using the \textit{execute} operation of the \textit{A3EDomain}. Since different domain can use different technology the \textit{InvocationResolver} binded to the domain will be use to actually invoke the function. 

Invocation resolver hides the burden of invocation technology details addressing the \textit{heterogeneity} property of the continuum. In the current implementation we created five types of \textit{InvocationResolver}s: one for executing a local javascript function, one for executing a local java function, one for Rest HTTP calls (this allows to support many FaaS platforms such as OpenWhisk, one specifically that exploit the AWS Lambda SDK and finally a local invocation resolver that is binded to the local domain that simply forward the invocation to the function's local \textit{InvocationResolver} (either the \textit{JavaInvocationResolver} or \textit{JSInvocationResolver}). 

Since each function has its own characteristic, for example different REST calls could have different \textit{content-type}s and different formatted input/output, we made each function able to modify the invocation with it is own applicative details. For this reason before invoking the actual function (e.g., performing the HTTP call) the \textit{InvocationResolver} creates a technology specific \textit{InvocationMean}. These components are wrappers to technology specific objects used by the invocation resolvers to build the call (e.g., the \textit{InvokeRequest\footnote{\url{http://docs.aws.amazon.com/AWSJavaSDK/latest/javadoc/com/amazonaws/services/lambda/model/InvokeRequest.html}}} object of the AWS Lambda SDK). As shown in Figure~\ref{fig:mobile-prototype} \textit{A3EFunction}s are \textit{InvocationMeanVisitor}s thus just before the performing the invocation the specific \textit{visit} method is call providing to the function a way to fill up the request for the specific type of call. 

To summarize, supporting the continuum with A3E in a client application consists in the following four steps\footnote{A simple example of use can be found at \url{https://github.com/gioenn/a3e-android/blob/master/README.md}}:

\begin{enumerate}
	\item Register static \textit{A3EDomain}s such as cloud endpoints.
	\item Create \textit{A3EFunction}s and override the specific methods to customize the requests for the technologies of interests 
	\item Register the functions
	\item Execute the functions in the continuum as a plain programmatic calls 
\end{enumerate}
