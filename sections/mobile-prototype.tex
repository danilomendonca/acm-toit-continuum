\subsection{Client-side Middleware}

%As described in Section~\ref{sec:proposal}, the CSM component interacts with DSM from different domains that could be either discoverable using a DNS-like mechanism or advertisement. 

The goal of the client-side middleware is to allow client applications to invoke A3-E microservices without knowing where they will actually be executed within the computing continuum: locally on the mobile device, in a local-edge server, in a mobile-edge server, or in the cloud. Its selection algorithm is a multi-objective function that takes into account the measured QoS and the requirements. The implementation we present\footnote{Documentation and source code are available at \url{https://github.com/deib-polimi/A3-E-CSM}} is in Java, so that it could be run on Android-based devices. However, it does not use Android-specific technology and can therefore easily be generalized to other mobile platforms. 

\begin{figure}[tbp]
	\includegraphics[width=1\textwidth]{figs/a3e-mobile-prototype}
	\caption{Client-side Middleware Architecture}
	\label{fig:mobile-prototype}
\end{figure}

Figure~\ref{fig:mobile-prototype} shows the high-level architecture of the client-side middleware. Client mobile applications interact with two components: \textit{A3EService}s and \textit{A3EFacade}. The former abstract the actual microservices to be invoked, while the latter is used to register and invoke the domain's microservices. 

An \textit{A3EService} is identified by a unique name and the url of its git code repository. It may also specify three types of requirements: 
%that corresponds to the name of the function asset that must be communicated to the continuum domains to be first acquired and then executed. Moreover each function must declare a set of non-functional QoS requirements that are organized in three types: 

\begin{enumerate}


	\item \textit{Location Requirement}s constrain where the microservice can be placed within the continuum, i.e., \textit{LOCAL}, \textit{LOCAL\_EDGE}, \textit{MOBILE\_EDGE}, or \textit{CLOUD} or a combination of the above; 


	\item a \textit{Latency Requirement} constrains network latency, i.e., \textit{ANY}, \textit{LOW} or \textit{VERY\_LOW}; and 
	

	\item a \textit{Computational Requirement} defines how relevant it is for a microservice to have fast computing, i.e., \textit{ANY}, \textit{FAST} or \textit{VERY\_ FAST}. 


\end{enumerate}

%Before becoming able for execution, the function must be registered using the \textit{A3EFacade}. 

%\begin{itemize}
	%\item \textit{Location Requirement}s are constrains over the continuum. A function can declare where it could be executed choosing a combination of three values: \textit{LOCAL}, \textit{EDGE} and \textit{CLOUD}. By default an \textit{A3EFunction} supports all three domain types but one can implement a function that, for example, cannot be executed locally thus only the \textit{EDGE} and \textit{CLOUD} requirements should be added.    
	%\item \textit{Latency Requirement} expresses how important for a function is to have a low networking latency. The default value is \textit{LOW} since the main motivation of the work is to support low-latency applications. An \textit{A3EFunction}  can  also state that this requirement should be \textit{ANY} or \textit{VERY LOW}. The lower the latency requested the more the networking latency will be considered important in the the domain selection procedure.
	%\item \textit{Computational Requirement} defines how relevant for a function is to have a fast computing processing. The predefined value is \textit{FAST}, since, again, the main targets of the approach are applications that requires fast request/response interactions. Similarly to the latency requirement two additional values are available:  \textit{ANY} or \textit{VERY FAST}. The higher  processing power is requested the more this metric will be considered important during the domain selection phase.
%\end{itemize}

%An \textit{A3EFunction} that support the \textit{LOCAL} location requirement should also define a local \textit{InvocationResolver}. As we are going to discuss later in this section, invocation resolvers deal with the invocation technology \textit{heterogenity}  of the continuum. For what regards the local domain we currently support the execution of native Java code and Javascript functions (that could be imported in the project as standard \textit{.js} files). For this purpose we created two \textit{InvocationResolver}s that can execute respectively Java or Javascript functions if the local domain is selected.

The \textit{A3EFacade} component wraps the \textit{A3EManager}, which manages all the registered \textit{A3EService}s. It consists of a MAPE control loop~\cite{kephart2003vision} for the discovery, identification and selection of domains, corresponding to the \textit{Awareness}, \textit{Acquisition}, and \textit{Allocation} phases of the A3-E model. 

%The loop manager consists of three main components called \textit{DiscoveryManager}, \textit{IdentificationManager} and \textit{SelectionManager}. 

Component \textit{DiscoveryManager} manages the discovery of domains. The local domain is registered when the middleware is created, while cloud domains are registered by the client application, and their endpoints must be known a-priori. Edge domains, on the other hand, are found at run time. Every time the IP address of the client device changes, the manager starts listening for UDP broadcast messages using component \textit{UDPDiscoverer}. If a message is received within a fixed timeout of $10$ seconds, the IP address of the sender is retrieved. A new \textit{A3EDomain} is then created in the middleware and added to a list of available domains. Each \textit{A3EDomain} is identified by a \textit{host} name (URI) and satisfies a specific \textit{LocationRequirement}. 

%Everytime the connection the domain is lost, the \textit{DiscoveryManager} updates its list of available domains.

 %Component \textit{InvocationResolver} is then used to actually invoke the micro-services. 

%An  \textit{A3EDomain} could be either \textit{static} or \textit{dynamic}. Static domains are added to the discovery manager at launch time, meaning that it is known a-priori that they will be available. Example of this are cloud and local domains. On contrary dynamic domains can be found only at runtime with appropriate technological protocols (such as DNS and advertising). Edge domains are not known a-priori thus they are considered dynamic. Static domains can be added by the client application using the \textit{registerDomain} operation provided by the \textit{A3EFacade} while dynamic domains are automatically discovered by the \textit{DiscoveryManager}.
 %\textit{A3EDomain}s  are identified by a \textit{host} name, that is a unique network identifier by which it is possible to interact with the domain using a network. Moreover, as shown in the figure, a domain is considered \textit{Pingable} that means that it is possible to check for its availability and measure its networking latency (in milliseconds). Each domain satisfy a \textit{LocationRequirement}, intuitively local domains satisfies the \textit{LOCAL} requirement, while the edge and cloud ones the \textit{EDGE} and \textit{CLOUD} respectively.  A domain also provides three additional operations: one to execute a function within the domain, one to retrieve the networking latency (its value is lazily updated after a \textit{ping}) and one to obtain its computational power. Finally a domain also uses an \textit{InvocationResolver} to actually invoke the function. 


%The output of this process is the list of the available domains. 
 %This component initializes the identification phase for each new available domain (e.g., domains discovered for the first time). 

As soon as a new domain is discovered, or registered, the \textit{IdentificationManager} requests to it all the registered microservices that are executable in that domain. This is done using component \textit{UDPIndentifier}, which communicates the microservices' unique name and repository address to the domain. A microservice is not ready to be executed in the domain until a UDP confirmation message is received meaning that it is ready for the engagement phase. 

\begin{algorithm}[b]
	\caption{A3E Selection Algorithm}
	\label{alg:selection}
	\begin{algorithmic}[1]
		
		\Function{selectDomain}{A3EService $microservice$, A3EDomain[] $identifiedDomains$}
		\State$scoreRange \gets 5$
		\State $maxLatency \gets \Call{computeMaximumLatency}{identifiedDomains}$
		\State $maxCpuPower \gets \Call{computeMaximumComputationalPower}{identifiedDomains}$
		\State $latencyWeight \gets microservice.getLatencyRequirement()$ 
		\State $cpuPowerWeight \gets microservice.getComputationalPowerRequirement()$ 
		\State $maxScore \gets 0$
		\State $selectedDomain \gets null$
		\ForAll{$domain \in identifiedDomains$ } 
		\State $latency \gets domain.getLatency()$ 
		\State $cpuPower \gets microservice.getComputationalPower()$ 
		\State $latencyScore \gets latencyWeight*((scoreRange-1)*(1 - latency/maxLatency)+1)$ 
		\State $cpuPowerScore \gets cpuPowerWeight*(scoreRange*(cpuPower/maxCpuPower))$
		\State $score \gets (latencyScore + cpuPowerScore) / (latencyWeight + cpuPowerWeight)$
		\If{$score \geq maxScore$} 
		\State $maxScore \gets score$
		\State $selectedDomain \gets domain$
		\EndIf
		\EndFor 
		\State \Return $selectedDomain$
		\EndFunction
	\end{algorithmic}
\end{algorithm}

\subsubsection{Selection Control Loop}

Component \textit{SelectionManager} is responsible for managing the selection control loop. The loop is performed by three sub-components called \textit{QoSMontior}, \textit{RequirementsAnalyzer} and \textit{DomainSelector}, and is activated once every $2$ seconds (by default).

The first sub-component retrieves the domains' computational power and network latency. The former is modeled as a fixed score ranging from $1$ to $5$\footnote{Labeling computational power is also common in the cloud where different tiers of virtual machines are available -- \url{https://aws.amazon.com/ec2/instance-types/}}. The local domain has a score of $1$, the edge domains have a score of $4$, while the cloud domains have a score of $5$. Note that since the cloud provides the illusion of infinite scalability it gets the maximum score, regardless of the VMs that are actually being used. More sophisticated approaches with dynamic scores taking into account the saturation of the domain or the device's battery level (in the case of a mobile domain), are considered as future work.  
 
The requirements analysis and the actual domain selection, perfomed by components \textit{RequirementsAnalyzer} and \textit{DomainSelector}, is described in Algorithm~\ref{alg:selection}. The procedure is activated for each registered microservice along with the identified domains (line $1$). The algorithm computes a score ranging from $0$ to $5$ (line $2$) for each domain and selects the highest one. The current implementation considers as quality metrics the networking latency and the computational power. The first step of the selection consists in computing the maximum latency and computational power (line $3$ and $4$) among the identified domains. This data, as already mentioned, is gathered during the previous phases. Then the weights assigned to latency and computational power are retrieved (lines $5$ and $6$). These weights correspond to the values associated to the \textit{LatencyRequirement} and \textit{ComputationalRequirement} of the microservice. The \textit{ANY} value corresponds to a weight of $0$, a latency requirement of \textit{LOW} and a computational power requirement of \textit{FAST} correspond to a weight equal to $1$, while a latency requirement of \textit{VERY\_LOW} and a computational power requirement of \textit{VERY\_FAST} correspond to a weight equal to $2$.

For each domain, the algorithm computes the overall score (line $9$ to $14$). The latency score is computed by normalizing the value retrieved at line $10$ with the maximum latency previously computed. The normalized value ranges from $0$ to $1$, the higher this value is the higher the latency. Since a higher score should mean lower latencies, the algorithm computes the complement of this value and adds $1$ to avoid scores that equal $0$. Finally, the latency score is computed to be in the range of $5$ and multiplied by the microservice's latency weight (line $12$). The computational power score is computed by normalizing the domain computational power retrieved at line $11$ with the maximum value across the identified domains. Then the score for this metric is computed to be in the range of $5$ and multiplied by its weight (line $13$). Finally the overall score is computed by performing a weighted average between the scores obtained by the domains for the two QoS metrics.

%Two considerations must be added for this algorithm. First, 
The procedure that we followed consists of an instantiation of the SMART decision process~\cite{Olson1996}, in which multiple competing QoS objectives are taken into account using the following formula:

\begin{equation}
Smart(c) = \frac{\sum_{i=1}^{n} QoSAtrribute_i(c)*weight_i}{\sum_{i=1}^{n}weight_i} \label{eq:smart}
\end{equation}

\noindent
where $c$ is a domain (see Figure~\ref{fig:domain-selection-loop}), the QoS attributes values are network latency and the computational processing time (thus $n$ is $2$), and their weights are represented by the aforementioned latency and computation requirements. Note that, when available, \textit{edge domains have the highest chances of being selected}, since they usually combine a low network latency and a medium-to-high computational power. Finally, each microservice is mapped to the domain that best satisfies its requirements. 

During the \textit{Engagement} phase the CSM can invoke a microservice using the \textit{invokeService} operation of the \textit{A3EFacade}. In addition to the \textit{A3EService}, the operation also expects a \textit{payload} (i.e., the function argument), and a \textit{callback} since  execution is always asynchronous. 

The \textit{A3EFacade} will retrieve the \textit{A3EDomain} that was selected for the microservice in the last control loop iteration. Since domains use heterogeneous technologies, the actual invocations are performed by the \textit{InvocationResolver} components that are bound to the different domains. 

The \textit{InvocationResolver} bound to edge and cloud domains will fire an HTTP request, while the \textit{InvocationREsolver} bound to a mobile domain will broadcast an Android event containing a request. Regardless, asynchronous \textit{callback}s are used to provide the results back to the client application. %\textit{InvocationResolver}s hide the technological details of the underlying domains. 
% bound to the domain will be used to actually \textit{execute} the function.  
%Since the control loop runs in a separated thread, the execution will not be affected if the selected domain changes during the invocation. After retrieving the best domain from the continuum, the \textit{A3EFacade} will execute the function by using the \textit{execute} operation of the \textit{A3EDomain}. Since each domains can use different technologies, the \textit{InvocationResolver} binded to the domain will be used to actually invoke the function. 
%addressing the \textit{heterogeneity} property of the continuum. 
In our current implementation of the client-side middleware we have three types of \textit{InvocationResolver}s: one for invoking microservices provided by a mobile domain, another for Rest HTTP calls (to support multiple FaaS platforms such as OpenWhisk\footnote{\url{https://github.com/apache/incubator-openwhisk/blob/master/docs/webactions.md}}), and one specifically designed for AWS Lambda\footnote{\url{https://aws.amazon.com/tools/\#sdk}}. %and finally a local invocation resolver that is binded to the local domain that simply forward the invocation to the function's local \textit{InvocationResolver} (either the \textit{JavaInvocationResolver} or \textit{JSInvocationResolver}).  


%Since each function has its own characteristics, for example different REST calls could have different \textit{content-type}s and different format for input/output, we made each function able to modify the invocation with it is own applicative details. For this reason before invoking the actual function (e.g., performing the HTTP call) the \textit{InvocationResolver} creates a technology specific \textit{InvocationMean}. These components are wrappers to technology specific objects used by the invocation resolvers to build the call (e.g., the \textit{InvokeRequest\footnote{\url{http://docs.aws.amazon.com/AWSJavaSDK/latest/javadoc/com/amazonaws/services/lambda/model/InvokeRequest.html}}} object of the AWS Lambda SDK). As shown in Figure~\ref{fig:mobile-prototype} \textit{A3EFunction}s are \textit{InvocationMeanVisitor}s. Thus, just before performing the invocation, the specific \textit{visit} method is call providing to the function a way to fill up the request for the specific type of call. 

%To summarize, supporting the continuum with A3E in a client application consists in the following four steps\footnote{A simple example of use can be found at \url{https://github.com/gioenn/a3e-android/blob/master/README.md}}:

%\begin{enumerate}
%	\item Register static \textit{A3EDomain}s such as cloud endpoints.
%	\item Create \textit{A3EFunction}s and override the specific methods to customize the requests for the technologies of interest
%	\item Register the functions
%	\item Execute the functions in the continuum as plain programmatic calls 
%\end{enumerate}