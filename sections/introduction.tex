\section{Introduction}

With the advent of Internet of Things, the evolution of mobile computing, and the emergence of real-time applications, the processing of an exponentially increasing volume of data must be performed in a timely fashion, i.e., with minimum latency. Despite the elasticity and vast computing power of existing cloud platforms, the access to these resources involves multiple hops of network communication, adding to the latency in which requisitions are processed. Such limitation has the following implications:

\begin{enumerate}

\item Cloud services may fail to satisfy the requirements of real-time and low-latency client applications; and

\item Offloading of delay-sensitive computation from devices with constrained resources to cloud servers (also known as Mobile Cloud Computing~\cite{}) is unlikely to work

\end{enumerate}

To reduce network latency, data processing must be performed closer to where it is produced and consumed. In accordance with this principle, the emerging paradigm of edge computing~\cite{} states that computing power should be pushed from centralized datacenters to the edge of the network. The realization of edge computing, however, still poses many challenges. 

%
Firstly, a highly distributed edge infrastructure is not expected to exhibit virtually unlimited resources as cloud datacenters. This limitation requires an efficient usage of edge resources and existing execution models based on virtualization and containerization, broadly adopted by cloud providers, may not be feasible with edge computing.

Secondly, in contrast with the high availability of cloud services covering very large areas from which clients requests are always expected, a fine grained coverage area of edge infrastructure would mean that many edge services would remain idle while there are no clients of these services around. Therefore, to optimize the usage of edge resources, the deployment of services to edge infrastructure should happen on-demand in an opportunistic fashion. 

%, accessible through well-known Internet names, edge services 

Finally, cloud servers should not be disregarded. First, because client applications may still rely on cloud backends for non delay-sensitive computation. Second, because edge servers may be unavailable from the current client location. Thus, client applications must rely on a runtime mechanism to discover and negotiate the use of nearby edge infrastructure. As alternative infrastructures may exhibit uneven loads, the decision of which to use, including the cloud, must take into account their current status as well as the client application requirements. 

%Finally, to avoid increasing the burden of application development, computation to be executed at the edge should follow, to the extent possible, a common architecture and implementation with respect to its cloud counterpart. The same applies to the computation that may be offloaded from resource constrained devices to nearby edge infrastructure.  

\subsection{Motivation}

\subsubsection{Realtime Applications}

The main motivation for displacing computation from cloud to edge infrastructure is to avoid network latency. In specific, real-time applications are the main candidates for benefiting of services deployed at proximal edge infrastructure. Among these, some have a higher degree of criticality with respect to latency and readiness of services (e.g., delay can have severe consequences to users), whereas others can greatly benefit from lower latencies (e.g., by improving the user experience).

As an example of a critical application, autonomous vehicles/drones...

In contrast, augmented reality (AR) is a type of application that would benefit from the low-latency of edge services. AR applications enrich the representation of the physical world with virtual elements like information about buildings and monuments or tips to guide users in the achievement of physical tasks. AR applications commonly depend on two key computational tasks: 1) extracting features from physical elements in the captured scene; and 2) matching these features against a feature database to obtain the corresponding information. 

With the advent of mobile computing,  mobile augmented reality (MAR) applications can be deployed to companion devices like smartphones and tablets. Nonetheless, MAR applications relying on data that is too large or too frequently updated to be ported to devices must rely on external servers. As MAR applications captures live representation of the physical world, this reality must be augmented at runtime, meaning data must be retrieved in a timely fashion. Due to network latency, a cloud-based solution tends to fail. Accordingly, feature extraction task should be deployed near to client devices, e.g.., to edge servers. 

\begin{figure}[htbp]
\centering
\subfloat[first caption.\label{fig:cloud-to-edge}]{\includegraphics[width=0.49\textwidth]{figs/cloud-to-edge.png}}\hfill
\subfloat[second caption.\label{fig:mobile-to-edge}] {\includegraphics[width=0.49\textwidth]{figs/mobile-to-edge.png}}\hfill

\caption{General caption.} \label{fig:1}
\end{figure}


\subsubsection{Mobile Computation Offloading}

In addition to the problem of network latency, mobile devices exhibit limitations that may further motivate the use of edge computing. 

For instance, some mobile applications rely on heavyweight tasks that can overstress the platform and limit the concurrent execution of other applications. Moreover, battery is a valuable resource that may be significantly affected by the kind of computation performed by mobile devices. 

In previous works~\cite{Mobile Cloud Computing}, this problem had been addressed with the offloading of mobile computation to cloud servers. This solution, however, is limited by network latency.

The paradigm of edge computing can be explored to mitigate the problems related to the resource limitations of mobile devices. For this, heavyweight computation from mobile applications could be offloaded to nearby edge servers. 

As an example, the previously mentioned feature extraction task from MAR applications is an example of a heavyweight computation based on image processing. Instead of performing it locally, mobile devices could offload it to nearby edge servers. 

\subsection{Serverless Computing}

Serverless computing emerged as a new execution model for cloud computing in which server management and capacity planning decisions are hidden from the software application engineers. In this model, serverless providers dynamically manage the allocation of machine resources. Multiple vendors are now offering serverless compute runtime and database services. 

Among its main advantages, the serverless model is considered to be cost-efficient in comparison to virtual machine and container-based provisioning models, which generally involve significant periods of underutilization or idle time. As it has been argued elsewhere~\cite{ESOCC'17}, a serverless architecture can be employed to enable low-latency applications to make use of edge computing computational resources in an efficient and scalable way. Additionally, mobile devices could make use of compute runtimes deployed at nearby edge servers to extend their capabilities. Notwithstanding the potential of such combination, a complete model for its realization is still missing. 

\subsection{Contributions of this Work}

In this work, we propose a unified model for the the provisioning of different kinds of edge infrastructure as a service. We also specify a reference architecture based on the paradigm of serverless computing. The proposed architecture is suitable for both cloud and edge infrastructures, reducing the burden of application development. Finally, we evaluated our model and architecture with two scenarios of edge computing: one in which edge servers are located at cellular infrastructure, and another in which edge servers are located indoors at an office building. The results showed the feasibility and scalability of providing edge infrastructure as a service.

TODOs: 
\begin{itemize}

\item add a critical application class example
\item introduce the concept of edge continuum

\end{itemize}

\subsection{Paper Organization}

This paper is organized as follows...




