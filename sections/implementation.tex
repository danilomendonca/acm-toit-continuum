%!TEX root = ../main.tex
% -*- root: ../main.tex -*-
\section{Implementation}\label{sec:implementation}

%As previously stated, in this paper we also present working prototypes of domain-side and client-side middlewares. In particular, the former consist of a mobile domain and a local-edge domain, while the latter consists of a Android mobile client device. 

To demonstrate the feasibility of our model
in managing the life-cycle of continuum applications
, we describe an implementation of A3-E. Due to its complexness, in this section we focus on the self-management loops handling Allocation from both provider and client viewpoints. %Nonetheless, a complete prototype was used for the evaluation in Sec.~\ref{sec:evaluation}.

\subsection{Domain Manager: Local-Edge}\label{sec:local-edge-domain-DSM}

%\subsubsection{Local-Edge Domains}

The domain manager prototype\footnote{Documentation and source code available at \url{https://github.com/deib-polimi/A3-E-DSM-local-edge/}} described herein focuses on a local-edge domain, and encompasses A3-E's Awareness and Acquisition. Both the Allocation and Engagement activities are delegated to the FaaS platform (OpenWhisk), which allocates $\mu$-services to its pool of containers and handles client application requests (fired by the middleware) by means of RESTful endpoints.

The local-edge manager prototype implements Awareness by broadcasting \textit{domain identification} UDP signals in a constant interval. A client device entering the network replies --- by means of a UDP unicast --- with the \textit{client identification} signal containing the $\mu$-services its client application requires, along with the respective repository from which $\mu$-service artifacts can be fetched during Acquisition (as described in Sec.~\ref{sec:A3-E-awareness}). 

For each identified $\mu$-service, the manager proceeds with Acquisition. Among the downloaded files, a descriptor provides instructions regarding installation (e.g., compilation of Java classes and required dependencies). In particular, the manager prototype relies on Gradle\footnote{https://gradle.org/}, a state-of-art build tool commonly employed with projects ranging from mobile applications to $\mu$-services.

Once downloaded and built, the edge domain Acquisition finishes with the deployment of $\mu$-service function(s) and dependences to OpenWhisk by means of its command line interface and the generation of a \textit{$\mu$-service acquired} signal with the same UDP unicast channel. In case of failure, the mobile middleware is informed with a \textit{$\mu$-service denied} signal (as described in Sec.~\ref{sec:A3-E-acquisition}).

\subsection{Domain Manager: Mobile Domain}\label{sec:mobile-domain-DSM}

Herein we describe a domain manager implementation for the Android platform\footnote{Documentation and source code available at \url{https://github.com/deib-polimi/A3-E-DSM-mobile/}}. 
%As previously explained in Sec.~\ref{sec:A3-E}, mobile domains are exempt of performing Acquisition, thus this section details the implementation of A3-E's Awareness and Engagement. 
%[Danilo] shall we divise the middleware from the mobile domain or are they packed together?
The resulting implementation was packaged as a module within the mobile middleware for Android platform (described in Sec.~\ref{sec:mobile_middleware}). 

The prototype implements Awareness by triggering a system-level \textit{domain identification} signal once it has been loaded by the middleware and by listening to a \textit{client identification} reply signal. 
In contrast with cloud and edge domains, 
%Instead of a repository, 
each \textit{$\mu$-service identified} signal contains the qualified name (e.g., the system path of a Java class implementing the static function), which is added to a \textit{service registry} implementing Acquisition. 

The prototype supports two types of $\mu$-service functions: Java functions, which are natively supported, and JavaScript functions, which require a JNI wrapper for their execution by the Android platform. Note that existing FaaS platforms support a variety of other languages and runtimes. More comprehensive implementations of the mobile domain can either make use of additional wrappers; developers may also provide native implementations of $\mu$-service functions for this domain. 

%for each $\mu$-service required by the application. 

Once a $\mu$-service function is registered, the mobile domain triggers a \textit{$\mu$-service acquired} event, which enables its selection by the middleware. During Engagement and upon the selection of this domain, \textit{$\mu$-service request} events are handled with the lookup of the corresponding function. Once found, the function is called with the parameters composing the original \textit{C-request}; when finished, a \textit{$\mu$-service response} signal containing the execution result is generated.

 %Decoupling between the client application and its mobile domain was achieved with the use of Android events triggered by the CSM (see Section~\ref{sec:CSM}) and handled by the DSM. 
%Each request contains the unique name of a microservice. 




\subsection{Mobile Middleware}~\label{sec:mobile_middleware}

The domain selection consists of a self-managing loop that: (i) monitors $\mu$-services provided by different domains in terms of QoS metrics; (ii) analyzes the best alternative satisfying requirements of the client application; and (iii) updates the domain selection for a given $\mu$-service. In specific, the analysis consists of a multi-attribute rating that takes into account the measured QoS attributes and application requirements.

In addition to A3-E's \textit{Location requirements}, the prototype considered two types of \textit{QoS requirements}:

{\small
\begin{enumerate}
	
	%	\item \textit{Location Requirement}s constrain where the $\mu$-service can be placed within the continuum, i.e., \textit{LOCAL}, \textit{LOCAL\_EDGE}, \textit{MOBILE\_EDGE}, or \textit{CLOUD} or a combination of the above; 
	
	\item a \textit{Latency Requirement} constrains network latency, i.e., \textit{ANY}, \textit{LOW} or \textit{VERY\_LOW}; and 
	
	\item a \textit{Computational Requirement} defines how relevant it is for a $\mu$-servi to have fast computing, i.e., \textit{ANY}, \textit{FAST} or \textit{VERY\_ FAST}. 
\end{enumerate}
}%

The latter is defined as a fixed score ranging from $1$ to $5$\footnote{Labeling computational power is also common in the cloud where different tiers of virtual machines are available -- \url{https://aws.amazon.com/ec2/instance-types/}}. By default, the mobile domain has a score of $1$, edge domains have a score of $4$, while cloud domains have a score of $5$. As cloud provides the illusion of infinite scalability it gets the maximum score, regardless of the VMs that are actually being used. 

%with dynamic scores taking into account the saturation of the domain or the device's battery level (in the case of a mobile domain).
\setlength{\textfloatsep}{5pt}% Remove \textfloatsep
{\scriptsize
\begin{algorithm}[h]
	\caption{A3E Selection Algorithm}
	\label{alg:selection}
	\begin{algorithmic}[1]		
		\Function{selectDomain}{A3EService service, A3EDomain[] $identifiedDomains$}
		\State$scoreRange \gets 5$
		\State $maxLatency \gets \Call{computeMaximumLatency}{identifiedDomains}$
		\State $maxCpuPower \gets \Call{computeMaximumComputationalPower}{identifiedDomains}$
		\State $latencyWeight \gets service.getLatencyRequirement()$ 
		\State $cpuPowerWeight \gets service.getComputationalPowerRequirement()$ 
		\State $maxScore \gets 0$
		\State $selectedDomain \gets null$
		\ForAll{$domain \in identifiedDomains$ } 
		\State $latency \gets domain.getLatency()$ 
		\State $cpuPower \gets service.getComputationalPower()$ 
		\State $latencyScore \gets latencyWeight*((scoreRange-1)*(1 - latency/maxLatency)+1)$ 
		\State $cpuPowerScore \gets cpuPowerWeight*(scoreRange*(cpuPower/maxCpuPower))$
		\State $score \gets (latencyScore + cpuPowerScore) / (latencyWeight + cpuPowerWeight)$
		\If{$score \geq maxScore$} 
		\State $maxScore \gets score$
		\State $selectedDomain \gets domain$
		\EndIf
		\EndFor 
		\State \Return $selectedDomain$
		\EndFunction
	\end{algorithmic}
\end{algorithm}
}%

%TODO [Danilo] improve this paragraph
Algorithm~\ref{alg:selection} describes the procedure employed in the $\mu$-service selection. The algorithm computes a score ranging from $0$ to $5$ (line $2$). First, it retrieves the maximum computational power and network latency from available domains (line $3$ and $4$). Then, it retrieves the weights assigned to each QoS metric (lines $5$ and $6$). These weights correspond to the values associated to the \textit{LatencyRequirement} and \textit{ComputationalRequirement} of the $\mu$-service. The \textit{ANY} value corresponds to a weight of $0$, a latency requirement of \textit{LOW} and a computational power requirement of \textit{FAST} correspond to a weight equal to $1$, while a latency requirement of \textit{VERY\_LOW} and a computational power requirement of \textit{VERY\_FAST} correspond to a weight equal to $2$. 

For each domain, the algorithm computes the overall score (line $9$ to $14$). The latency score is computed by normalizing the value retrieved at line $10$ with the maximum latency previously computed. The normalized value ranges from $0$ to $1$, the higher this value is the higher the latency. Since a higher score should mean lower latencies, the algorithm computes the complement of this value and adds $1$ to avoid scores equal to $0$. The latency score is computed to be between $1$ and $5$, and multiplied by the $\mu$-service's latency weight (line $12$). The computational power score is computed by normalizing the domain computational power retrieved at line $11$ with the maximum value across the identified domains. Again, the score for this metric is computed to be between $1$ to $5$ and multiplied by its weight (line $13$). Finally, the overall score is the weighted average between the scores obtained by the domains for the two QoS metrics.

%Two considerations must be added for this algorithm. First, 
Algorithm~\ref{alg:selection} is an instantiation of the SMART~\cite{Olson1996}, in which multiple competing QoS attributes are taken into account using the following formula:
{\small
\begin{equation}
Smart(p) = \frac{\sum_{u=1}^{U} actual_{u}(p)*weight_u}{\sum_{u=1}^{U}weight_u} \label{eq:smart}
\end{equation}
}%

\noindent
where $p$ is a domain, the considered QoS attributes are network latency and the computational processing time (thus $U = 2$), and their weights are represented by the aforementioned latency and computation requirements. Note that, when available, edge domains have the highest chances of being selected, since they usually combine a low network latency and a medium-to-high computational power. 

%Accordingly, each $\mu$-service is mapped to the domain that best satisfies its requirements.

%Last but not least, during the \textit{Engagement} phase the CSM handles C-requests triggered by the client application for a specific $\mu$-service in the continuum and invokes the domain previously selected. Domains are bound to an invocation resolver: edge and cloud domains resolvers fire an HTTP request, while the resolver bound to a mobile domain will broadcast an Android event containing the request along with a callback. In particular, this broadcast is handled by the mobile domain manager.