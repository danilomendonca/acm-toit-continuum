\section{Conclusions and Future Work}\label{sec:conclusions}

Due to its \textit{automation} capabilities and \textit{efficiency}, which complement those of FaaS, A3-E provides a suitable approach for the realization of the mobile-, edge-, and cloud-computing continuum. It overcomes the domains' heterogeneity thanks to stateless functions exposed as microservices, and tackles both the placement of computation along the continuum and their selection by clients.

A3-E has been assessed with experiments that demonstrate its ability to model and enable a mobile image recognition application that opportunistically exploits the continuum. The experiments also show a substantial reduction of both latency and battery when edge services are used instead of cloud or local ones. 

Ongoing and future work include intra and inter-domain advanced resource management along the continuum. 
Several approaches in literature tackle the problem of single domain (i.e., a cloud provider) dynamic resource allocation by using heuristics~\cite{dustdar0}, artificial intelligence~\cite{ia1} or queue theory~\cite{queue1}. 
Resource provisioning for continuum applications poses new challenges: one has to consider their low-latency nature and their highly dynamic workload (e.g., from users that quickly enter and leave a particular edge area). To tackle this problem we are working on a self-management loop based on lightweight control theoretical algorithms (as proposed in~\cite{Quatrocchi2016discrete})) that could handle the fast allocation of containers employed by continuum $\mu$-service instances. Indeed, compared to virtual machines containers can be booted in few seconds, therefore controllers are able to change the resource allocation with a very fast period (i.e., in the order of few seconds). 

Inter-domain resource management should consider not only the heterogeneity (by leveraging existing algorithms~\cite{Tarneberg2017}), but also that different domains may be managed by different providers (e.g., local-edge and mobile-edge) and not be able to communicate with one another. We are also working on more robust multi-objective algorithms for the client-side service selection. 
Last but not least, 
%as mobile devices are also domains, 
we would like to explore scenarios of computation offloading in which devices featuring more computational resources can be seen as mobile domains for constrained devices in case nearby edge domains are not available and the latency to the cloud is prohibitive.