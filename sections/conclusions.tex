\section{Conclusions and Future Work}\label{sec:conclusions}

In this paper we addressed the realization of a computational continuum formed by cloud, edge, and mobile computing with A3-E. Due to its particular characteristics of \textit{automation} and \textit{efficiency}, which inherit and complement those of FaaS, A3-E provides a suitable approach for the realization of edge computing, the key piece in the continuum. It also overcomes the heterogeneity of cloud, edge and mobile computing with stateless functions that may be implemented with a common language and ported to different parts of the continuum. Finally, it tackles the placement of services along the continuum with a dual process of allocation involving both clients and providers.

A3-E has been assessed with experiments that demonstrated the feasibility of A3-E in modeling and enabling a mobile agumented-reality application to consume services dynamically selected from local, edge, and cloud providers based on their availability and its requirements for latency. The experiments also showed a substantial reduction of both latency and battery when edge services were used instead of cloud and local services. 

Ongoing and future work include resource management along the continuum, i.e., inter-domain resource management. This should consider not only the heterogeneity across the continuum (by leveraging existing algorithms~\cite{Tarneberg2017}), but also that different domains may be managed by different providers (e.g., local-edge and mobile-edge) and not be able to communicate to each other. Additionally, as mobile devices are also domains, we are exploring scenarios for device-to-device computation offloading~\cite{Mendonca2016A3droid}, e.g., when nearby edge domains are not able to fulfill application requirements and the latency to the cloud is prohibitive.
