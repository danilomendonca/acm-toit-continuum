\section{Conclusions and Future Work}\label{sec:conclusions}

In this paper we addressed the realization of a computational continuum formed by cloud, edge, and mobile computing with A3-E. Due to its particular characteristics of \textit{automation} and \textit{efficiency}, which inherit and complement those of FaaS, A3-E provides a suitable approach for the realization of edge computing, the key piece in the continuum. It also overcomes the heterogeneity of cloud, edge and mobile computing with stateless functions that may be implemented with a common language and ported to different parts of the continuum. Finally, it tackles the placement of services along the continuum with a dual process of allocation involving both clients and providers.

A3-E has been assessed with experiments that demonstrated the feasibility of A3-E in modeling and enabling a mobile agumented-reality application to consume services dynamically selected from local, edge, and cloud providers based on their availability and its requirements for latency. The experiments also showed a substantial reduction of both latency and battery when edge services were used instead of cloud and local services. 

Ongoing and future work include simulations with multiple clients using several applications with different requirements deployed along the continuum. We are also working on deploying edge domains with different specifications in terms of computational power and latency (e.g., deploying on bare metal servers and Polimi private cloud), to better reflect the heterogeneity of the continuum. Finally, as mobile devices are also domains, we are exploring scenarios for device-to-device computation offloading.

%TODO we can mention device-to-device communication, simulations with many clients sharing a piece of the continuum, heterogeneous specifications for the edge, deploying more and more applications to the continuum (not only the AR one).

 