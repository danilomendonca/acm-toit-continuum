\subsection{Domain Manager: Local-Edge}\label{sec:local-edge-domain-DSM}

%\subsubsection{Local-Edge Domains}

The domain manager prototype\footnote{Documentation and source code available at \url{https://github.com/deib-polimi/A3-E-DSM-local-edge/}} described herein focuses on a local-edge domain, and encompasses A3-E's Awareness and Acquisition. Both the Allocation and Engagement activities are delegated to the FaaS platform (OpenWhisk), which allocates $\mu$-services to its pool of containers and handles client application requests (fired by the middleware) by means of RESTful endpoints.

The local-edge manager prototype implements Awareness by broadcasting \textit{domain identification} UDP signals in a constant interval. A client device entering the network replies --- by means of a UDP unicast --- with the \textit{client identification} signal containing the $\mu$-services its client application requires, along with the respective repository from which $\mu$-service artifacts can be fetched during Acquisition (as described in Sec.~\ref{sec:A3-E-awareness}). 

For each identified $\mu$-service, the manager proceeds with Acquisition. Among the downloaded files, a descriptor provides instructions regarding installation (e.g., compilation of Java classes and required dependencies). In particular, the manager prototype relies on Gradle~\footnote{https://gradle.org/}, a state-of-art build tool commonly employed with projects ranging from mobile applications to $\mu$-services.

Once downloaded and built, the edge domain Acquisition finishes with the deployment of $\mu$-service function(s) and dependences to OpenWhisk by means of its command line interface and the generation of a \textit{$\mu$-service acquired} signal back to the client. In case of failure, the mobile middleware is informed with a \textit{$\mu$-service denied} signal (as described in Sec.~\ref{sec:A3-E-acquisition}).

\subsection{Domain Manager: Mobile}\label{sec:mobile-domain-DSM}

Herein we describe a domain manager implementation for the Android platform\footnote{Documentation and source code available at \url{https://github.com/deib-polimi/A3-E-DSM-mobile/}}. 
%As previously explained in Sec.~\ref{sec:A3-E}, mobile domains are exempt of performing Acquisition, thus this section details the implementation of A3-E's Awareness and Engagement. 
%[Danilo] shall we divise the middleware from the mobile domain or are they packed together?
The resulting implementation was packaged as a module within the mobile middleware for Android platform (described in Sec.~\ref{sec:mobile_middleware}). 

The mobile domain manager implements Awareness by triggering a system-level \textit{domain identification} signal once it has been loaded by the middleware and by listening to a \textit{client identification} reply signal. In contrast with cloud and edge domains, the \textit{client identification} consists of the qualified name of $\mu$-service function(s) (e.g., the system path of a Java class implementing the static function). The latter is added to a service registry kept by the domain manager.

The prototype supports two types of $\mu$-service functions: Java functions, which are natively supported, and JavaScript functions, which require a JNI wrapper for their execution by the Android application. Note that current FaaS platforms support a variety of languages and runtimes, including Java and JavaScript functions. Nonetheless, mobile domains can either make use of additional wrappers or rely on a native implementation of $\mu$-service function(s).  

Once a $\mu$-service is successfully registered, the mobile domain finishes Acquisition with a \textit{$\mu$-service acquired} signal. During Engagement and upon selection of this domain by the mobile middleware, \textit{$\mu$-service request} signals are handled with the lookup of the corresponding function. Once found, the function is called with the parameters composing the request. Finally, a \textit{request response} signal is triggered with the execution result.

 %Decoupling between the client application and its mobile domain was achieved with the use of Android events triggered by the CSM (see Section~\ref{sec:CSM}) and handled by the DSM. 
%Each request contains the unique name of a microservice. 


